\documentclass[12pt]{article}

\usepackage[margin=1in]{geometry}
\usepackage{amsmath}
\usepackage{lipsum}
\usepackage{listings}
\usepackage{xcolor}
\usepackage{graphicx}
\graphicspath{{/home/bakunowski/QueenMary/FundamentalsOfDsp/Lab1/plots/}}

\definecolor{codegreen}{rgb}{0,0.6,0}
\definecolor{codegray}{rgb}{0.5,0.5,0.5}
\definecolor{codepurple}{rgb}{0.58,0,0.82}
\definecolor{backcolour}{rgb}{0.95,0.95,0.92}

\lstdefinestyle{mystyle}{
    backgroundcolor=\color{backcolour},
    commentstyle=\color{codegreen},
    keywordstyle=\color{magenta},
    numberstyle=\tiny\color{codegray},
    stringstyle=\color{codepurple},
    basicstyle=\ttfamily\footnotesize,
    breakatwhitespace=false,
    breaklines=true,
    captionpos=b,
    keepspaces=true,
    numbers=left,
    numbersep=5pt,
    showspaces=false,
    showstringspaces=false,
    showtabs=false,
    tabsize=2
}

\setlength{\parindent}{0em}
\setlength{\parskip}{1em}

% change font
%\usepackage[default]{sourcesanspro}
%\usepackage[T1]{fontenc}
% helvetica
% \usepackage[scaled]{helvet}
% \renewcommand\familydefault{\sfdefault}
% \usepackage[T1]{fontenc}

\lstset{style=mystyle}

\begin{document}

\begin{flushleft}
\normalsize
Karol Bakunowski \hfill 11 March 2020\\
\vspace{5 mm}
\Large
Assignment 1: Dynamic Range Compressor\\
\normalsize
\end{flushleft}

\section{Introduction}

A multiband dynamic range compressor VST plugin was developed using the JUCE
platform. In this report, the discussion of the implementation, as well as the
analysis of it's output are analysed.

\section{Design and Implementation}

The three band compressor implementation presented in this report was chosen to
demonstrate the implementation of a passband crossover, as well as maintaining
a certain level of computational complexity.

\subsection{Compressor}

% demo code was used as the basis, from book

% how the knee operates

% throw some equations in here?

\subsection{Crossover}

% the crossover was produced using the juce iir coeff. and iirfilter libraries

% talk about the linkwitz riley filter (4th order)

% quote juce website

\subsection{GUI}

% describe parameters here

\section{Analysis}

% to visualise the output of the multi band, a logarithmic sine wave sweep was passed as the input

% show a spectrogram and how plug in did

% sine sweep passed through the compressor
    % control case
    % three different settings - for each compressor

\section{Evaluation}

% shown that it performs as expected
% separate into distinct bands and apply different levels of compression


\end{document}
